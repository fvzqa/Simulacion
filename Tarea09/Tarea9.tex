\documentclass[12pt,letterpaper]{article}
\usepackage[utf8]{inputenc}
\usepackage{tikz}
\usetikzlibrary{trees}
\usepackage[spanish, es-nodecimaldot]{babel}
\usepackage{amsmath}
\usepackage{color}
\usepackage{algorithm}
\usepackage[noend]{algpseudocode}
\renewcommand{\algorithmicrequire}{\textbf{Entrada:}}
\renewcommand{\algorithmicensure}{\textbf{Salida:}}
\usepackage{subcaption}
\usepackage{amsfonts}
\usepackage{hyperref}
 \hypersetup{
     colorlinks=true,
     linkcolor=blue,
     filecolor=blue,
     citecolor = blue,      
     urlcolor=cyan,
     }
\usepackage{amssymb}
\usepackage{listings}
\usepackage{color}

\definecolor{mygreen}{rgb}{0,0.6,0}
\definecolor{mygray}{rgb}{0.5,0.5,0.5}
\definecolor{mymauve}{rgb}{0.58,0,0.82}

\lstset{ 
  backgroundcolor=\color{white},   % choose the background color; you must add \usepackage{color} or \usepackage{xcolor}; should come as last argument
  basicstyle=\footnotesize,        % the size of the fonts that are used for the code
  breakatwhitespace=false,         % sets if automatic breaks should only happen at whitespace
  breaklines=true,                 % sets automatic line breaking
  captionpos=b,                    % sets the caption-position to bottom
  commentstyle=\color{mygreen},    % comment style
  deletekeywords={...},            % if you want to delete keywords from the given language
  escapeinside={\%*}{*)},          % if you want to add LaTeX within your code
  extendedchars=true,              % lets you use non-ASCII characters; for 8-bits encodings only, does not work with UTF-8
  firstnumber=1,                % start line enumeration with line 1000
  frame=single,	                   % adds a frame around the code
  keepspaces=true,                 % keeps spaces in text, useful for keeping indentation of code (possibly needs columns=flexible)
  keywordstyle=\color{blue},       % keyword style
  language=Octave,                 % the language of the code
  morekeywords={*,...},            % if you want to add more keywords to the set
  numbers=none,                    % where to put the line-numbers; possible values are (none, left, right)
  numbersep=5pt,                   % how far the line-numbers are from the code
  numberstyle=\tiny\color{mygray}, % the style that is used for the line-numbers
  rulecolor=\color{black},         % if not set, the frame-color may be changed on line-breaks within not-black text (e.g. comments (green here))
  showspaces=false,                % show spaces everywhere adding particular underscores; it overrides 'showstringspaces'
  showstringspaces=false,          % underline spaces within strings only
  showtabs=false,                  % show tabs within strings adding particular underscores
  stepnumber=2,                    % the step between two line-numbers. If it's 1, each line will be numbered
  stringstyle=\color{mymauve},     % string literal style
  tabsize=2,	                   % sets default tabsize to 2 spaces
  title=\lstname                   % show the filename of files included with \lstinputlisting; also try caption instead of title
}

\usepackage{amsthm}
\newtheorem{theorem}{Teorema}

\usepackage{graphicx}
\usepackage[inner=1.5 cm, outer = 1.5 cm, top=1 cm, bottom = 1.5 cm]{geometry}
\setlength{\parskip}{3mm}
\title{\textsc{Práctica 9: interacción entre partículas}}
\author{\textsc{Fabiola Vázquez}}
\renewcommand{\lstlistingname}{Código}
\setlength{\parindent}{0cm}

\begin{document}
\maketitle

\hrule
\section{Introducción}
En esta práctica \cite{elisapractica9} se tiene una cantidad \texttt{n} de partículas en un cuadro unitario, cada una de ellas posee una carga y una masa. El objetivo es verificar gráficamente que existe una relación entre los factores velocidad, carga y masa de las partículas. El experimento se lleva a cabo en el software R \cite{R} en un cuaderno de Jupyter \cite{jupyter}.

\section{Experimento}
Las partículas son posicionadas normalmente al azar en el cuadro unitario. Además, éstas poseen una carga que causa fuerzas de repulsión y atracción con las demás partículas si dos partículas tienen el mismo signo o diferente, respectivamente. Se desea añadir otro atributo a las partículas que afecte dicha interacción, este atributo es la masa y se considera el tamaño de ésta en $\{0.2, 0.4, 0.6, 0.8, 1.0\}.$ 

En la figura \ref{im} se muestra la interacción entre 100 partículas, el atributo de la masa es representado con círculos de diferente tamaño para mostrar las diferencia entre las masas de las partículas. Esta interacción se puede apreciar de una mejor manera en una \href{https://github.com/fvzqa/Simulacion/blob/master/Tarea09/Images/p9.gif}{animación}.
 \begin{figure}
 	\centering 
 	\begin{subfigure}[b]{0.3\linewidth}
 		\includegraphics[width=\linewidth]{p9_t000.png} 		
 		\caption{Estado inicial.}
 		 		\label{1}
 	\end{subfigure}  \hfill
 	\begin{subfigure}[b]{0.3\linewidth}
 		\includegraphics[width=\linewidth]{p9_t050.png} 		
 		\caption{Paso 50.}
 		\label{2}
 	\end{subfigure} \hfill
 	 	\begin{subfigure}[b]{0.3\linewidth}
 		\includegraphics[width=\linewidth]{p9_t100.png} 		
 		\caption{Paso 100.}
 		\label{3}
 	\end{subfigure}
 	 	\caption{Interacción entre 100 partículas.} 
 	 		\label{im}
\end{figure} 

\begin{figure}
\centering
\includegraphics[width=\linewidth]{scatterplot.png} 		
 		\caption{Gráficas de dispersión a pares.}
 		\label{4}
\end{figure}

En el experimento se considera la interacción de 500 partículas, como se quiere que la masa afecte la velocidad, la fuerza calculada mediante las cargas de las partículas se divide entre la masa que posee la partícula. Con esto se quiere lograr que partículas con mayor masa se muevan más despacio que las que tienen menor masa. Durante cada iteración se calcula la distancia que ha recorrido cada partícula para saber su velocidad promedio, toda la información se almacena en un \texttt{data.frame}, del cual se muestra un fragmento en el cuadro \ref{data}.

Con la información obtenida, se realizan diversos gráficos de dispersión a pares de los factores: velocidad, masa, posición inicial y carga. La figura \ref{4} muestra dichos gráficos y como se aprecia la carga y la masa afectan a la velocidad.

Se realiza una prueba de correlación para comprobar si la masa de las partículas afecta su velocidad y se obtiene un valor $p \approx 2.2\times 10^{-16}$ el cual es menor que 0.05 por lo que se concluye que existe una correlación entre los factores masa y velocidad. Dado que el coeficiente de correlación es, aproximadamente, $-0.6922$, a mayor masa de la partícula menor es su velocidad, como se quería al inicio.
\begin{table}
\centering
\caption{Fragmento de los datos.}
\begin{tabular}{rrrrrr}
  \hline
Partícula & Posición $x$ & Posición $y$ & Carga & Masa &Velocidad\\ 
  \hline
1 & 0.4136 & 0.8686 & -0.7423 & 1.0 & 0.0018 \\ 
2 & 0.6571 & 0.2439 & 1.4867 & 0.6 & 0.0046 \\ 
3 & 0.3473 & 0.9417 & -0.8000 & 0.2 & 0.0090\\ 
4 & 0.5780 & 0.8338 & -0.8613 & 1.0 & 0.0020 \\
5 & 0.5364 & 0.8442 & 0.5731 & 0.2 & 0.0080 \\
6 & 0.9836 & 0.9878 & 0.3748 & 1.0 & 0.0015\\
\hline
\end{tabular}
\label{data}
\end{table}
\bibliographystyle{plain} 
\bibliography{Referencias}


\end{document} 